The goal of a stream cipher is to securely send a stream of information.
Understanding a stream as a sequence of bits, bytes or larger words.
As in all cryptography we assume that the channel used is insecure.
In order to transmit some plaintext stream we encrypt it block by block to a cyphertext stream.
This stream is then transmited over the channel.
The receptor of this stream, performs the reverse operation of the ecryption operation.
Doing so, it obtains the original plaintext.
One important requirement of stream ciphers is that we can encrypt and decrypt \textbf{on-line}.
On-line meanining that the reception of the entire message or ciphertext is not required to start the procedures.

The most basic stream cipher is Vernam's cipher where the stream is combined with a pseudorandom stream of data.
They are combined by using the boolean exclusive or (XOR) function (see Equation~\ref{eq:stream_encryption}).
Ideally this pseudorandom stream of data would be a random number as long as the message.
For practical reasons, this is impossible as we would need a key as long as all the messages we want to send.
In the practical approach the key stream is generated (pseudorandom stream).
It is created from a long term key $k$ and internal state $st$ and a transition function $F$.
Equation~\ref{eq:key_stream_generation} shows how to combine them to obtain the key stream $r$.


\begin{equation}
c_n = m_n \oplus r_n
\label{eq:stream_encryption}
\end{equation}

\begin{equation}
(r_n, st_n) = F(st_{n-1}, k)
\label{eq:key_stream_generation}
\end{equation}

A symmetric key encryption is called \textbf{synchronous} if the key stream can be generated independently from the message and cyphertext.
This is the case for all stream ciphers based on Vernam's cipher.

Due to the recursive nature of the key stream generation algorithm an base case or initial state $st_0$ has to be included.
This initial state can be transmitted in plain text over the insecure channel, because the attacker does not have access to $k$.
He therefore is not capable of generating subsecuent blocks of the key stream.
An important property about this initial state is that it must be different between encryptions.
If the same initial state is used the property in Equation~\ref{eq:nonce_repetition_keystream} holds.
This would allow the attacker to forge a valid ciphertext.

\begin{equation}
m_n \oplus m_n' = c_n \oplus c_n'
\label{eq:nonce_repetition_keystream}
\end{equation}

Regarding transmission error, a bit flip does not compromise the rest of the stream decryption.
However, if the syncronization is lost between the sender and the receiver it causes a permanent decryption failure.
This comes from the fact that if the internal states do not match the key stream will never match again.

In Vernam based ciphers the security is based on the \textbf{unpredictability} of the key stream when the key is unknown.
The algorithm is totally secure if the bits have the same probability and are independent of each other.

An simple and \textbf{insecure} example of key stream is a linear feedback shift register (LFSR).
A LFSR is a sequence of $m$ cells and a linear feedback function that pushes a new bit into the left cell.
When a new bit is pushed, the rest of bits are pushed to the right.
The right most bit is given a output to be used in the key stream.

LFSR by themself are not secure as their linear nature makes them vulnerable to ciptoanalysis.
They can be made more secure by implementing non-linear combinations, combining non-linearly the output of several registers or by using an irregular clock.
This updated methods are used in application such as GSM or Bluetooth.

In most applications more complex, non-linear ciphers are used to improve security.
One really important example for cyptography history is RC4 as it was consifered insecure after the attack to WEP in 2001.
The security flaw was mainly due to the key secheduling.

RC4 is a byte based key stream generated with a simple finite state machine.
It has several deviations from Equation~\ref{eq:key_stream_generation} as the function $F$ does not use the key.
The $F$ function only uses the internal state (see Equation~\ref{eq:rc4_F})

\begin{equation}
(r_n, st_n) = F(st_{n-1})
\label{eq:rc4_F}
\end{equation}

The key $k$ and a nonce are used to generate the intial state.
For it a key expansion function $F_0$ is used.
The key is expanded into an internal state of 2048 bits, representing a map $S$.

\begin{equation}
st_0 = F_0(k)
\label{eq:rc4_F0}
\end{equation}

The algorithm use for expanding the key is the presented in the following Python snippet.
It then intialises with the index and then performs pseudorandom swaps.

\inputminted{python}{src/rc4-init.py}

The state update function just performs a swap with a non linear relationship and returns a byte at a random index. 
It is shown in the following Python snippet.

\inputminted{python}{src/rc4.py}

On important issue with RC4 is that there is not a specification on how to add a nonce to the encryption.
It is important in case we want to reuse the long term key.
In poor implementations like WEP the nonce is prepended to the long term key.
Making the inital part of the key stream predicatable once an attacker learns some pairs of cyphertext and plaintext.

A widely used stream cipher nowadays is ChaCha.
It is used by applications such as Google TLS, OpenSSH and the Linux /dev/urandom.

The design of this stream cipher is completly different from RC4.
It does not use a finite state machine and every key stream block can be computed independently of the others.
A nonce $u$ (unique for each encrypted message) and a counter $i$ (unique for each block in the message) are used as inputs.
This two values alongside the long term key $k$ are used to compute the key stream block $k_i$ (see Equation~\ref{eq:chacha}).

\begin{equation}
  k_i = F(k, u, i)
  \label{eq:chacha}
\end{equation}

The key stream is computed in blocks of 512 bits, flow a long term key of 256 bits.
Each block is initialized with a 128-bit constant, the long term key, a 64-bit sequential block and a 64-bit nonce.
Then in the algorithm, the block contents is suffled with a function.

Previously we have described some cryptosytems as being secure or not but no formal definition was given.
We define the level of security of a cryptosystem by the type of attacks it can resist.
Defining an attack means giving the attacker a goal and some ressources.

The most simple attack defined this ways is recovering the key with the description of the cipher and som cipertexts of unknown plaintexts.
The attacker is considered succesful if it correctly guesses the long term key.
Another attacker goals could be predicting portions of the key, learning som partial information of a plaintext or letting the recipient accept a corrupted message as valid.

In practical scenarios the attacker can have additional resource, like lerning some pairs of his chosing or not.
He can also have access to other sources like data of the communication channel, timing information or power consuption.
The attacker can also just evesdrop the communication or even modify it.
Also limitations on the computational power and attack time are also considered in the attack models.
