The basis behind block ciphers is that the length of plaintexts, ciphertexts and keys is fixed in advance.
As the input and output are fixed, the encryption can be defined as a function.
This function maps the key and message space to the ciphertext space (see Equation~\ref{eq:ske_block_enc_space}).
In the same way, decription maps the key and cipher space to the message space (see Equation~\ref{eq:ske_block_dec_space}).
This only holds as long as the encryption is deterministic, which is most of the cases.

\begin{equation}
  E:K \times M \rightarrow C
\label{eq:ske_block_enc_space}
\end{equation}

\begin{equation}
  D:K \times C \rightarrow M
\label{eq:ske_block_dec_space}
\end{equation}

In order to simplify notation the following notation simplifications will be taken into account.
In it the key is added as a subindex to remove it from the argument list.

\begin{equation*}
  E_k(m)=E(k, m)
\end{equation*}

\begin{equation*}
  D_k(m)=D(k, m)
\end{equation*}

$E_k$ must be an injective map meaning that two inputs can not be mapped to the same output.
If it is not the case a ciphertext can be decrypted to two different plaintexts causing problems in decryption.
Therefore a block cipher can be understood as a collection of injective maps indexed by the secret key.
In an ideal cipher this set of map is the full set of injective maps from M to C.
This ideal cipher is totally impractical but it is still useful as an idealized model of a block cipher.

In practical cippher the map $E_k$ behaves like a random injective map.
Normally the designs are iterative, a series of operations are defined and repeated several times.
This ciphers must follow to important properties.
The first one is \textbf{confusion} where every bit in the cipher must depend on several bits of the key.
The second one is \textbf{diffusion} where flipping a single plaintext bit must change half of the ciphertext bits.
For every iteration or round, we use a different subkey obtained from the main key in the \textbf{key scheuling} procedure.
A round combines different permutation and substitution operations.

DES only now considered insecure by it short key length is an example of block cipher.
It uses a key length of 56-bits with a round subkey of 48-bits.
Each block has a size of 64-bits and is passed through 16 rounds.

The rounds used by DES are based on Feistel networks.
This rounds first split the input message into the left part and right part (see Equation~\ref{eq:feistel0}).
It then obtains the next block by permutating both blocks and xoring the right side combined with key trough $F$ (see Equation~\ref{eq:feistel1}).
Finally, the ciphertext is the output of the 16th round (see Equation~\ref{eq:feistel2}).


\begin{equation}
  (L_0, R_0) = m
\label{eq:feistel0}
\end{equation}

\begin{equation}
  (L_{n+1}, R_{n+1}) = (R_n, L_n \oplus F(k_n, R_n))
\label{eq:feistel1}
\end{equation}

\begin{equation}
  c = (R_{16}, L_{16})
\label{eq:feistel2}
\end{equation}

The $k$ with subindex are the different round subkeys obtained from the original key.
If we perform the same procedure reversing the subkey order we obtain the decryption function.

An important step for this network to be successful is the function $F$.
This function first expands the half block from 32-bits to 48-bits and xors it with the round subkey.
It is then divided into 8 6-bit pieces and each one replaced by a 4-bit word using the correesponding substitution table.
The 8 4-bit words are glued together into the 32-bit half block.
Finally we apply the permutation P to the 32 bits.

DES is considered insecure because the key only contains 56-bits, which means it can be bruteforced by an attacker.
Triple DES (or 3DES) allows using longer keys and is still considered secure.
In triple DES the middle encryption is in reallity a decryption for back compatibility reasons.
If the 3 keys are the same in triple DES it is equivalent to regular DES.

AES is widely used nowadays in applications like TLS, SSL, disk encryption, compression tools and Signal.
It uses a key length of 128, 192 or 256 bits with a round subkey of 128 bits.
The block length used in AES is 128-bits and the number of rounds depends on the key size (10, 12 or 14).

A round of AES arranges the block as a 4 by 4 matrix of bytes.
Each byte in the block is replaced by another according to a substitution box S.
The i-th row is rotated i position to the left breaking the colomn struture of the matrix.
A $F_{2^8}$ linear map is applied to the matrix columns as a multiplication be a particular polynomial.
Each row is interpreted as a polynomial in a way that each byte is the coeficient of the polynomial.
The resulting block is xored with the round subkey.

The key addition is performed before starting the first round, and the las round does not mix the columns.
All the steps are invertible and their inverses are used during decryption.

With the descryption of blocks ciphers and their real world example we can see that encrypting only 128-bits can be very limiting.
Furthermore, the encryption is deterministic, the same message sent twice has the same ciphertext.
For this reasons, block cipher are intended to be used in specific \textbf{mode of operation}.
This modes of operation specify how to deal with messages longer than one block and randomization.

The \textbf{Electronic Code Book} (ECB) is the simplest way to encrypt messages of arbitrary length, but it is completly insecure.
They simple split the data into blocks and each block is encrypted sparately with the same key (see Equation~\ref{eq:ecb}).

\begin{equation}
  c_i = E_k(m_i)
  \label{eq:ecb}
\end{equation}

The main problem is that the equality of two plaintext blocks in the sequence implied the equality of the corresponding ciphertext blocks.
In the case of network packets, if two packets share the same header, their headers will share cipher blocks.
There is a semantic leakage and a lack of randomization.
A classical example is that encrypting an image will yield the same image but with different colored squares.

The \textbf{Cipher Block Chaining} Mode (CBC) uses an initialization value (IV).
It allows that two encyptions of the same message with the same key provide different ciphertexts.
This initializaton value must not be reused, because it then defeats the purpose.
A value that is only use once is refered to as a \textbf{nonce}.

The encryption is performed as specified in Equation~\ref{eq:cbc_enc}.
Each block encrypt the message xored with the previous encryted block.
For the first block, it is xored with the encrypted IV.

\begin{equation}
  c_i = E_k(c_{i-1} \oplus m_i)
  \label{eq:cbc_enc}
\end{equation}

\begin{equation}
  c_0 = E_k(IV)
  \label{eq:cbc_iv}
\end{equation}

As for the decryption operation, it simple inverts the order of the operations.
It first decrypts and then xors with the previous ciphertext (see Equation~\ref{eq:cbc_dec}).

\begin{equation}
  m_i = D_k(c_i) \oplus c_{i-1}
  \label{eq:cbc_dec}
\end{equation}

If there is an error during transmission for $c_i$, it would only affect to the decryption of two consecutive messages.
The error would affect $m_i$ and also would affect $m_{i+1}$ as it uses $c_i$ for decryption.
Due to the nature of encryption it is not parallelizable, but for the same reason than previously decryption is.
This is due to the fact that only the previous ciphertext is required, not the entire chain or other plaintext.

The blocksize of the block cipher used in CBC mode must be large enough to minimize the probability of collision.
It must avoid that two ciphertext bloocks are equal.
If we have a collision (see Equation~\ref{eq:cbc_col}), then Equation~\ref{eq:cbc_col_aenc} holds true.
As encryption is injective, if they are equal after encryption they must be before (see Equation~\ref{eq:cbc_col_benc}).
By using properties of the xor operation we can obtain the result in Equation~\ref{eq:cbc_col_res}.
This is clearly a vulnerability, for this reason collision must be highly improbable.

\begin{equation}
  c_i = c_j
  \label{eq:cbc_col}
\end{equation}

\begin{equation}
  E_k(c_{i-1} \oplus m_i)  = E_k(c_{j-1} \oplus m_j)
  \label{eq:cbc_col_aenc}
\end{equation}

\begin{equation}
  c_{i-1} \oplus m_i  = c_{j-1} \oplus m_j
  \label{eq:cbc_col_benc}
\end{equation}

\begin{equation}
  m_i \oplus m_j = c_{i-1} \oplus c_{j-1}
  \label{eq:cbc_col_res}
\end{equation}

The \textbf{Cipher Feedback} Mode (CFB) has some similarities with CBC but has a main advantage over it.
It only uses the encryption function of the block cipher, we can use block ciphers without decryption function.
In order to encypt it xors the rencryption of the previous ciphertext with the message (see Equation~\ref{cfb_enc}).

\begin{equation}
  c_i = E_k(c_{i-1}) \oplus m_i
  \label{eq:cfb_enc}
\end{equation}

As for decryption, the operation is identical to encryption but prociding the ciphertext instead of plaintext.

\begin{equation}
  m_i = E_k(c_{i-1}) \oplus c_i
  \label{eq:cfb_enc}
\end{equation}

If two ciphertext blocks collide a similar problem as for CBC apears where properties of plaintexts is leaked.
For this reason, collition resistance in the block cipher is a requirement for using CFB Mode.

The \textbf{Output Feedback} (OFB) Mode works as a synchronous stream cipher.
It uses an IV and a the secret key to generate a long pseudorandom sequence (see Equations~\ref{eq:ofb_keystream_init}~and~\ref{eq:ofb_keystream}).
This long sequence is the xored with the message stream (see Equation~\ref{eq:ofb_enc}).
Like in CFB Mode only the encryption module is required as the operations for encryption and decription are identical (see Equation~\ref{eq:ofb_dec}).

\begin{equation}
  r_i = E_k(r_{i-1})
  \label{eq:ofb_keystream}
\end{equation}

\begin{equation}
  r_0 = IV
  \label{eq:ofb_keystream_init}
\end{equation}

\begin{equation}
  c_i = r_i \oplus m_i
  \label{eq:ofb_enc}
\end{equation}

\begin{equation}
  m_i = r_i \oplus c_i
  \label{eq:ofb_dec}
\end{equation}

Transmission errors do not propagate, if the IV is correctly transmitted. A corrupted IV causes decryption errors in all blocks.

The \textbf{Counter} Mode (CTR) is similar to OFB but with an easier to produce sequence of blocks, depending on the stream.
The operation between the IV and index usually is concatenation but xor can also be done (see Equation~\ref{eq:ofb_keystream}).
The encryption (Equation~\ref{eq:ofb_enc}) and decryption (Equation~\ref{eq:ofb_dec}) are simply a stream cipher.

\begin{equation}
  s_i = IV || i
  \label{eq:ofb_keystream}
\end{equation}

\begin{equation}
  c_i = E_k(s_i) \oplus m_i
  \label{eq:ofb_enc}
\end{equation}

\begin{equation}
  m_i = E_k(s_i) \oplus c_i
  \label{eq:ofb_dec}
\end{equation}

Bothe encryption and decryption of blocks can be done independently, and therefore, both procedures are fully parallelizable.
As in OFB, only IV transmission failure causes failure in all the blocks.
