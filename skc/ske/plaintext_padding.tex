In practical application the plaintext length is not an exact multiple of the block size of a block cipher.
For this reason, it is necessary to add the missing bits to complete the last block.
The proces of adding this missing bits is called padding.

Padding must be done in a way that no ambiguity is introduced.
Adding zeros after the message would not be enough unless we assume the last bit of the message to be one.
A simple padding scheme is appending a 1-bit and zero or more 0-bits.
For a message that exactly matches the block size we must add an extra block with just padding.
This is a common pattern amongst padding schemes.
During decryption we must search for the last bit with 1 and delete everything from there forwards.
