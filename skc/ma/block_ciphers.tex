A block cipher is a natural construction for combining all the bits of a message with a (short) secret key.
They are combined in such a way that the key is not exposed, even if some pairs are revealed to the attacker.
Therefore, some MAC schemes are based on block ciphers operating in chaining modes like CBC.

A characteristic of the MAC that can be exploited is that MAC is only for integrity not messagbe recovery.
Meaning that the MAC output can be a lossful transformation, allowing for the tag to be shorter than the message.
Usually if a block cipher is used, the MAC has the size of a single block.

In 1989, \textbf{CBC-MAC} was created.
It consists in taking the last block of encryption of the message operating in CBC mode and with a zero initial IV.
CBC-MAC is proven secure for messages with a fixed length that is an exact multiple of the block length.

A successful attack against CBC-MAC for a given block cipher, implies there exists also a successful attack against the block cipher.
In order to prove this statement a formal description of both attack models is required.
For CBC-MAC, the attack chooses two messages og the same length and asks for the MAC of one of them.
The attacker wins if it correctly guesses which message has been used to compute the MAC.
For the block cipher, the attackers asks for encyption of as many blocks as it wants.
Then it eirher receives the encrypted blocks or just random elements.
The attacker succeeds if it correctly huesses whether it is receiving real encryption or random elements.
This proof requires a minimum block length in roder to ensure a low collision probability of ciphertexts.

CBC-MAC is proven to be insecure if applied to messages of arbitrary lengths.
Proof in the laboratory exercises.

The \textbf{One-Key MAC} (2003) is a modification of CBC-MAC to make it secure for messages of variable length.
In OMAC, the last block in the message is padded if necessary with the previously introduced padding scheme.
It then is tweaked with extra keys (Equation~\ref{eq:ok_mac_end}) before performing the last encryption operation in CBC-MAC.
In Equation~\ref{eq:ok_mac_end} $m'_n$ represents the padded block and $k'$ the subkey obtained from $k$.
This subkey varies depending on whether the last block has been padded or not.

\begin{equation}
  c_i = E_k(c_{i-1} \oplus m_i)
  \label{eq:ok_mac_base}
\end{equation}

\begin{equation}
  c_n = E_k(c_{i-1} \oplus m'_n \oplus k')
  \label{eq:ok_mac_end}
\end{equation}
