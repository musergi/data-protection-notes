A block cipher is a natural construction for combining all the bits of a message with a (short) secret key.
They are combined in such a way that the key is not exposed, even if some pairs are revealed to the attacker.
Therefore, some MAC schemes are based on block ciphers operating in chaining modes like CBC.

A characteristic of the MAC that can be exploited is that MAC is only for integrity not messagbe recovery.
Meaning that the MAC output can be a lossful transformation, allowing for the tag to be shorter than the message.
Usually if a block cipher is used, the MAC has the size of a single block.

In 1989, \textbf{CBC-MAC} was created.
It consists in taking the last block of encryption of the message operating in CBC mode and with a zero initial IV.
CBC-MAC is proven secure for messages with a fixed length that is an exact multiple of the block length.
