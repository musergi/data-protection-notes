Configentialy is not the only security property of a communication system that must be guaranteed.
An attacker coyuld try to modify the content of a message without being detected.
He could even try to convice the recipient that the corrupted message was created by the sender.
\textbf{Data integrity} is the property that the recipient of a message can detect (with high probability) tampering.
Understanding as tampering, any manipulation during the transmission of the message.

A commonly used way to ensure integrity to a message is appending a \textbf{message authentication code} (MAC) to it.

In a MAC scheme a long term secret key $k$ is shared between a sender and a recipient.
This key is used in two operations, the first being generating the tag or MAC.
The second is the verification with a given pair message and tag that the message is not modified.
Both oprations must follow Equation~\ref{eq:mac_prot} in order to ensure correctness.

\begin{equation}
  Ver(k, m, MAC(k, m)) = 1
  \label{eq:mac_prop}
\end{equation}

The authentication procedure starts with the sneder generating a tag $t$ (see Equation~\ref{eq:mac_gen}).
The sender then sends to the recipient the pair $(m, t)$ through an insecure channel.
The recipient receives a pair $(m', t')$ that can be different from the sent pair.
The recipient can then check the integrity by using the comparison in Equation~\ref{eq:mac_ver}.

\begin{equation}
  t = MAC(k, m)
  \label{eq:mac_gen}
\end{equation}

\begin{equation}
  Ver(k, m) ?= 1
  \label{eq:mac_ver}
\end{equation}

MAC could be a probabilistic algorith, meaning there would be several valid tags for a given message and key.
However, normally it is deterministic and the verification function just recalculates the tag and compares it with the received.
In this case, correctness is trivial.

A MAC system is secure if an attacker is unable to forge a valid pair of message and tag.
But, like in cryptography, different attack scenarios can be defined, giving several ressources and goals to the attacker.

In the simplest case the attacker knows the algorithm and some valid pairs from messages recorded.
The attacker wins if he is able to recover the key.
Other goals could be \textbf{universal forgery}, the attacker can forge a valid tag for any message without the key.
Another could be \textbf{existential forgery}, the attacker can forge a tag for a particular message chosen by him.
This message is not within the known message and tag pairs.
The ressources for this attack can be some valid pairs for messages chosen by the attacker or not.

A correct label attached to a message not only gives the guarantees that the massage has not been modified.
It also ensure that the sender computed the tag, as he is the only one other than the recipient knowing the key.

Perfect schemes don't exist as with enough message and tag pairs most are broken.
For this reason, in security proofs we limit the amount of pairs available.
