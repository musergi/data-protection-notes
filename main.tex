\documentclass[10pt]{article}
\usepackage[USenglish]{babel}
\usepackage[useregional]{datetime2}
\usepackage{lipsum}
\usepackage{xcolor}
\usepackage{amsmath}
\DTMlangsetup[en-US]{showdayofmonth=false}

\newcommand{\rfcId}{DPROT}
\newcommand{\rfcTitle}{Notes}
\newcommand{\rfcAuthor}{Sergi Mus}
\newcommand{\rfcDate}{\today}
\newcommand{\rfcInstitution}{UPC}

% CODE
\usepackage{minted}
\definecolor{LightGray}{gray}{0.8}
\setminted[python]{frame=lines, framesep=2mm, bgcolor=LightGray}

% TABLE OF CONTENT
\usepackage{tocloft}
\renewcommand{\cftsecleader}{\cftdotfill{\cftdotsep}}
\renewcommand{\cftsubsecleader}{\cftdotfill{\cftdotsep}}
\renewcommand{\cftsubsubsecleader}{\cftdotfill{\cftdotsep}}
\renewcommand{\contentsname}{Table of Content}

% MARGINS
\usepackage{titlesec}
\titlelabel{\thetitle.\quad}
\usepackage{geometry} 
\geometry{
	a4paper,
	left=30mm,
	top=30mm,
	bottom=30mm,
	right=30mm
}
\setlength{\leftskip}{17pt}

% HEADER AND FOOTER
\usepackage{lastpage}
\usepackage{fancyhdr}
\pagestyle{fancyplain}
\fancyhead{}
\fancyfoot{}
\fancyhead[L]{\rfcId}
\fancyhead[C]{\rfcTitle}
\fancyhead[R]{\rfcDate}
\fancyfoot[L]{\rfcAuthor} 
\fancyfoot[C]{\rfcTitle} 
\fancyfoot[R]{[Page \thepage] \\} 
\renewcommand{\headrulewidth}{0pt} 
\renewcommand{\footrulewidth}{0pt} 
\setlength{\headheight}{13.6pt}

% FIRST PAGE
\usepackage{multicol}

% FONT
\usepackage{inconsolata}
\renewcommand{\familydefault}{\ttdefault}

\begin{document}

\begin{multicols}{2}
	\begin{flushleft}
		Internet Engineering Task Force \\
		DRAFT \rfcId
	\end{flushleft}
\columnbreak
	\begin{flushright}
		\rfcAuthor \\
		\rfcInstitution \\
		\rfcDate
	\end{flushright}
\end{multicols}

\vspace{1in} { \center \rfcTitle \\ } \vspace{1in}

\begin{abstract}
  This document contains the notes for the subject of Data Protection.
  In it we study in depth cryptography both symetric and assymetric cryptography. 
\end{abstract}
\pagebreak

\tableofcontents
\pagebreak

\section{Symmetric Key Cryptography}{
  \subsection{Symmetric Key Encryption}{
    \subsubsection{Stream Ciphers}{
      The goal of a stream cipher is to securely send a stream of information.
Understanding a stream as a sequence of bits, bytes or larger words.
As in all cryptography we assume that the channel used is insecure.
In order to transmit some plaintext stream we encrypt it block by block to a cyphertext stream.
This stream is then transmited over the channel.
The receptor of this stream, performs the reverse operation of the ecryption operation.
Doing so, it obtains the original plaintext.
One important requirement of stream ciphers is that we can encrypt and decrypt \textbf{on-line}.
On-line meanining that the reception of the entire message or ciphertext is not required to start the procedures.

The most basic stream cipher is Vernam's cipher where the stream is combined with a pseudorandom stream of data.
They are combined by using the boolean exclusive or (XOR) function (see Equation~\ref{eq:stream_encryption}).
Ideally this pseudorandom stream of data would be a random number as long as the message.
For practical reasons, this is impossible as we would need a key as long as all the messages we want to send.
In the practical approach the key stream is generated (pseudorandom stream).
It is created from a long term key $k$ and internal state $st$ and a transition function $F$.
Equation~\ref{eq:key_stream_generation} shows how to combine them to obtain the key stream $r$.


\begin{equation}
c_n = m_n \oplus r_n
\label{eq:stream_encryption}
\end{equation}

\begin{equation}
(r_n, st_n) = F(st_{n-1}, k)
\label{eq:key_stream_generation}
\end{equation}

A symmetric key encryption is called \textbf{synchronous} if the key stream can be generated independently from the message and cyphertext.
This is the case for all stream ciphers based on Vernam's cipher.

Due to the recursive nature of the key stream generation algorithm an base case or initial state $st_0$ has to be included.
This initial state can be transmitted in plain text over the insecure channel, because the attacker does not have access to $k$.
He therefore is not capable of generating subsecuent blocks of the key stream.
An important property about this initial state is that it must be different between encryptions.
If the same initial state is used the property in Equation~\ref{eq:nonce_repetition_keystream} holds.
This would allow the attacker to forge a valid ciphertext.

\begin{equation}
m_n \oplus m_n' = c_n \oplus c_n'
\label{eq:nonce_repetition_keystream}
\end{equation}

Regarding transmission error, a bit flip does not compromise the rest of the stream decryption.
However, if the syncronization is lost between the sender and the receiver it causes a permanent decryption failure.
This comes from the fact that if the internal states do not match the key stream will never match again.

In Vernam based ciphers the security is based on the \textbf{unpredictability} of the key stream when the key is unknown.
The algorithm is totally secure if the bits have the same probability and are independent of each other.

An simple and \textbf{insecure} example of key stream is a linear feedback shift register (LFSR).
A LFSR is a sequence of $m$ cells and a linear feedback function that pushes a new bit into the left cell.
When a new bit is pushed, the rest of bits are pushed to the right.
The right most bit is given a output to be used in the key stream.

LFSR by themself are not secure as their linear nature makes them vulnerable to ciptoanalysis.
They can be made more secure by implementing non-linear combinations, combining non-linearly the output of several registers or by using an irregular clock.
This updated methods are used in application such as GSM or Bluetooth.

In most applications more complex, non-linear ciphers are used to improve security.
One really important example for cyptography history is RC4 as it was consifered insecure after the attack to WEP in 2001.
The security flaw was mainly due to the key secheduling.

RC4 is a byte based key stream generated with a simple finite state machine.
It has several deviations from Equation~\ref{eq:key_stream_generation} as the function $F$ does not use the key.
The $F$ function only uses the internal state (see Equation~\ref{eq:rc4_F})

\begin{equation}
(r_n, st_n) = F(st_{n-1})
\label{eq:rc4_F}
\end{equation}

The key $k$ and a nonce are used to generate the intial state.
For it a key expansion function $F_0$ is used.
The key is expanded into an internal state of 2048 bits, representing a map $S$.

\begin{equation}
st_0 = F_0(k)
\label{eq:rc4_F0}
\end{equation}

The algorithm use for expanding the key is the presented in the following Python snippet.
It then intialises with the index and then performs pseudorandom swaps.

\inputminted{python}{src/rc4-init.py}

The state update function just performs a swap with a non linear relationship and returns a byte at a random index. 
It is shown in the following Python snippet.

\inputminted{python}{src/rc4.py}

On important issue with RC4 is that there is not a specification on how to add a nonce to the encryption.
It is important in case we want to reuse the long term key.
In poor implementations like WEP the nonce is prepended to the long term key.
Making the inital part of the key stream predicatable once an attacker learns some pairs of cyphertext and plaintext.

A widely used stream cipher nowadays is ChaCha.
It is used by applications such as Google TLS, OpenSSH and the Linux /dev/urandom.

The design of this stream cipher is completly different from RC4.
It does not use a finite state machine and every key stream block can be computed independently of the others.
A nonce $u$ (unique for each encrypted message) and a counter $i$ (unique for each block in the message) are used as inputs.
This two values alongside the long term key $k$ are used to compute the key stream block $k_i$ (see Equation~\ref{eq:chacha}).

\begin{equation}
  k_i = F(k, u, i)
  \label{eq:chacha}
\end{equation}

The key stream is computed in blocks of 512 bits, flow a long term key of 256 bits.
Each block is initialized with a 128-bit constant, the long term key, a 64-bit sequential block and a 64-bit nonce.
Then in the algorithm, the block contents is suffled with a function.

Previously we have described some cryptosytems as being secure or not but no formal definition was given.
We define the level of security of a cryptosystem by the type of attacks it can resist.
Defining an attack means giving the attacker a goal and some ressources.

The most simple attack defined this ways is recovering the key with the description of the cipher and som cipertexts of unknown plaintexts.
The attacker is considered succesful if it correctly guesses the long term key.
Another attacker goals could be predicting portions of the key, learning som partial information of a plaintext or letting the recipient accept a corrupted message as valid.

In practical scenarios the attacker can have additional resource, like lerning some pairs of his chosing or not.
He can also have access to other sources like data of the communication channel, timing information or power consuption.
The attacker can also just evesdrop the communication or even modify it.
Also limitations on the computational power and attack time are also considered in the attack models.

    }
    \subsubsection{Block Ciphers}{
      The basis behind block ciphers is that the length of plaintexts, ciphertexts and keys is fixed in advance.
As the input and output are fixed, the encryption can be defined as a function.
This function maps the key and message space to the ciphertext space (see Equation~\ref{eq:ske_block_enc_space}).
In the same way, decription maps the key and cipher space to the message space (see Equation~\ref{eq:ske_block_dec_space}).
This only holds as long as the encryption is deterministic, which is most of the cases.

\begin{equation}
  E:K \times M \rightarrow C
\label{eq:ske_block_enc_space}
\end{equation}

\begin{equation}
  D:K \times C \rightarrow M
\label{eq:ske_block_dec_space}
\end{equation}

In order to simplify notation the following notation simplifications will be taken into account.
In it the key is added as a subindex to remove it from the argument list.

\begin{equation*}
  E_k(m)=E(k, m)
\end{equation*}

\begin{equation*}
  D_k(m)=D(k, m)
\end{equation*}

$E_k$ must be an injective map meaning that two inputs can not be mapped to the same output.
If it is not the case a ciphertext can be decrypted to two different plaintexts causing problems in decryption.
Therefore a block cipher can be understood as a collection of injective maps indexed by the secret key.
In an ideal cipher this set of map is the full set of injective maps from M to C.
This ideal cipher is totally impractical but it is still useful as an idealized model of a block cipher.

In practical cippher the map $E_k$ behaves like a random injective map.
Normally the designs are iterative, a series of operations are defined and repeated several times.
This ciphers must follow to important properties.
The first one is \textbf{confusion} where every bit in the cipher must depend on several bits of the key.
The second one is \textbf{diffusion} where flipping a single plaintext bit must change half of the ciphertext bits.
For every iteration or round, we use a different subkey obtained from the main key in the \textbf{key scheuling} procedure.
A round combines different permutation and substitution operations.

DES only now considered insecure by it short key length is an example of block cipher.
It uses a key length of 56-bits with a round subkey of 48-bits.
Each block has a size of 64-bits and is passed through 16 rounds.

The rounds used by DES are based on Feistel networks.
This rounds first split the input message into the left part and right part (see Equation~\ref{eq:feistel0}).
It then obtains the next block by permutating both blocks and xoring the right side combined with key trough $F$ (see Equation~\ref{eq:feistel1}).
Finally, the ciphertext is the output of the 16th round (see Equation~\ref{eq:feistel2}).


\begin{equation}
  (L_0, R_0) = m
\label{eq:feistel0}
\end{equation}

\begin{equation}
  (L_{n+1}, R_{n+1}) = (R_n, L_n \oplus F(k_n, R_n))
\label{eq:feistel1}
\end{equation}

\begin{equation}
  c = (R_{16}, L_{16})
\label{eq:feistel2}
\end{equation}

The $k$ with subindex are the different round subkeys obtained from the original key.
If we perform the same procedure reversing the subkey order we obtain the decryption function.

An important step for this network to be successful is the function $F$.
This function first expands the half block from 32-bits to 48-bits and xors it with the round subkey.
It is then divided into 8 6-bit pieces and each one replaced by a 4-bit word using the correesponding substitution table.
The 8 4-bit words are glued together into the 32-bit half block.
Finally we apply the permutation P to the 32 bits.

DES is considered insecure because the key only contains 56-bits, which means it can be bruteforced by an attacker.
Triple DES (or 3DES) allows using longer keys and is still considered secure.
In triple DES the middle encryption is in reallity a decryption for back compatibility reasons.
If the 3 keys are the same in triple DES it is equivalent to regular DES.

AES is widely used nowadays in applications like TLS, SSL, disk encryption, compression tools and Signal.
It uses a key length of 128, 192 or 256 bits with a round subkey of 128 bits.
The block length used in AES is 128-bits and the number of rounds depends on the key size (10, 12 or 14).

A round of AES arranges the block as a 4 by 4 matrix of bytes.
Each byte in the block is replaced by another according to a substitution box S.
The i-th row is rotated i position to the left breaking the colomn struture of the matrix.
A $F_{2^8}$ linear map is applied to the matrix columns as a multiplication be a particular polynomial.
Each row is interpreted as a polynomial in a way that each byte is the coeficient of the polynomial.
The resulting block is xored with the round subkey.

The key addition is performed before starting the first round, and the las round does not mix the columns.
All the steps are invertible and their inverses are used during decryption.

With the descryption of blocks ciphers and their real world example we can see that encrypting only 128-bits can be very limiting.
Furthermore, the encryption is deterministic, the same message sent twice has the same ciphertext.
For this reasons, block cipher are intended to be used in specific \textbf{mode of operation}.
This modes of operation specify how to deal with messages longer than one block and randomization.

The \textbf{Electronic Code Book} (ECB) is the simplest way to encrypt messages of arbitrary length, but it is completly insecure.
They simple split the data into blocks and each block is encrypted sparately with the same key (see Equation~\ref{eq:ecb}).

\begin{equation}
  c_i = E_k(m_i)
  \label{eq:ecb}
\end{equation}

The main problem is that the equality of two plaintext blocks in the sequence implied the equality of the corresponding ciphertext blocks.
In the case of network packets, if two packets share the same header, their headers will share cipher blocks.
There is a semantic leakage and a lack of randomization.
A classical example is that encrypting an image will yield the same image but with different colored squares.

The \textbf{Cipher Block Chaining} Mode (CBC) uses an initialization value (IV).
It allows that two encyptions of the same message with the same key provide different ciphertexts.
This initializaton value must not be reused, because it then defeats the purpose.
A value that is only use once is refered to as a \textbf{nonce}.

The encryption is performed as specified in Equation~\ref{eq:cbc_enc}.
Each block encrypt the message xored with the previous encryted block.
For the first block, it is xored with the encrypted IV.

\begin{equation}
  c_i = E_k(c_{i-1} \oplus m_i)
  \label{eq:cbc_enc}
\end{equation}

\begin{equation}
  c_0 = E_k(IV)
  \label{eq:cbc_iv}
\end{equation}

As for the decryption operation, it simple inverts the order of the operations.
It first decrypts and then xors with the previous ciphertext (see Equation~\ref{eq:cbc_dec}).

\begin{equation}
  m_i = D_k(c_i) \oplus c_{i-1}
  \label{eq:cbc_dec}
\end{equation}

If there is an error during transmission for $c_i$, it would only affect to the decryption of two consecutive messages.
The error would affect $m_i$ and also would affect $m_{i+1}$ as it uses $c_i$ for decryption.
Due to the nature of encryption it is not parallelizable, but for the same reason than previously decryption is.
This is due to the fact that only the previous ciphertext is required, not the entire chain or other plaintext.

The blocksize of the block cipher used in CBC mode must be large enough to minimize the probability of collision.
It must avoid that two ciphertext bloocks are equal.
If we have a collision (see Equation~\ref{eq:cbc_col}), then Equation~\ref{eq:cbc_col_aenc} holds true.
As encryption is injective, if they are equal after encryption they must be before (see Equation~\ref{eq:cbc_col_benc}).
By using properties of the xor operation we can obtain the result in Equation~\ref{eq:cbc_col_res}.
This is clearly a vulnerability, for this reason collision must be highly improbable.

\begin{equation}
  c_i = c_j
  \label{eq:cbc_col}
\end{equation}

\begin{equation}
  E_k(c_{i-1} \oplus m_i)  = E_k(c_{j-1} \oplus m_j)
  \label{eq:cbc_col_aenc}
\end{equation}

\begin{equation}
  c_{i-1} \oplus m_i  = c_{j-1} \oplus m_j
  \label{eq:cbc_col_benc}
\end{equation}

\begin{equation}
  m_i \oplus m_j = c_{i-1} \oplus c_{j-1}
  \label{eq:cbc_col_res}
\end{equation}

The \textbf{Cipher Feedback} Mode (CFB) has some similarities with CBC but has a main advantage over it.
It only uses the encryption function of the block cipher, we can use block ciphers without decryption function.
In order to encypt it xors the rencryption of the previous ciphertext with the message (see Equation~\ref{cfb_enc}).

\begin{equation}
  c_i = E_k(c_{i-1}) \oplus m_i
  \label{eq:cfb_enc}
\end{equation}

As for decryption, the operation is identical to encryption but prociding the ciphertext instead of plaintext.

\begin{equation}
  m_i = E_k(c_{i-1}) \oplus c_i
  \label{eq:cfb_enc}
\end{equation}

If two ciphertext blocks collide a similar problem as for CBC apears where properties of plaintexts is leaked.
For this reason, collition resistance in the block cipher is a requirement for using CFB Mode.

The \textbf{Output Feedback} (OFB) Mode works as a synchronous stream cipher.
It uses an IV and a the secret key to generate a long pseudorandom sequence (see Equations~\ref{eq:ofb_keystream_init}~and~\ref{eq:ofb_keystream}).
This long sequence is the xored with the message stream (see Equation~\ref{eq:ofb_enc}).
Like in CFB Mode only the encryption module is required as the operations for encryption and decription are identical (see Equation~\ref{eq:ofb_dec}).

\begin{equation}
  r_i = E_k(r_{i-1})
  \label{eq:ofb_keystream}
\end{equation}

\begin{equation}
  r_0 = IV
  \label{eq:ofb_keystream_init}
\end{equation}

\begin{equation}
  c_i = r_i \oplus m_i
  \label{eq:ofb_enc}
\end{equation}

\begin{equation}
  m_i = r_i \oplus c_i
  \label{eq:ofb_dec}
\end{equation}

Transmission errors do not propagate, if the IV is correctly transmitted. A corrupted IV causes decryption errors in all blocks.

The \textbf{Counter} Mode (CTR) is similar to OFB but with an easier to produce sequence of blocks, depending on the stream.
The operation between the IV and index usually is concatenation but xor can also be done (see Equation~\ref{eq:ofb_keystream}).
The encryption (Equation~\ref{eq:ofb_enc}) and decryption (Equation~\ref{eq:ofb_dec}) are simply a stream cipher.

\begin{equation}
  s_i = IV || i
  \label{eq:ofb_keystream}
\end{equation}

\begin{equation}
  c_i = E_k(s_i) \oplus m_i
  \label{eq:ofb_enc}
\end{equation}

\begin{equation}
  m_i = E_k(s_i) \oplus c_i
  \label{eq:ofb_dec}
\end{equation}

Bothe encryption and decryption of blocks can be done independently, and therefore, both procedures are fully parallelizable.
As in OFB, only IV transmission failure causes failure in all the blocks.

    }
    \subsubsection{Self-Synchronizing Stream Ciphers}{
      Block ciphers used in OFB or CTR modes are actually synchronous block-oriented stream ciphers.
In them the encrypted symbols are not single bits but blocks.

\textbf{Self-synchronizing} stream ciphers solve the desynchronization problem.
They alow recovering the normal decryption operation after receiving a number of correct cipher stream consecutive symbols after the wrong onces.
A block cipher operating in CBC or CFB mode can recover from transmission errors after receiving two correct cipher blocks.

    }
    \subsubsection{Plaintext Padding}{
      In practical application the plaintext length is not an exact multiple of the block size of a block cipher.
For this reason, it is necessary to add the missing bits to complete the last block.
The proces of adding this missing bits is called padding.

Padding must be done in a way that no ambiguity is introduced.
Adding zeros after the message would not be enough unless we assume the last bit of the message to be one.
A simple padding scheme is appending a 1-bit and zero or more 0-bits.
For a message that exactly matches the block size we must add an extra block with just padding.
This is a common pattern amongst padding schemes.
During decryption we must search for the last bit with 1 and delete everything from there forwards.

    }
  }
  \subsection{Message authentication}{
    Configentialy is not the only security property of a communication system that must be guaranteed.
An attacker coyuld try to modify the content of a message without being detected.
He could even try to convice the recipient that the corrupted message was created by the sender.
\textbf{Data integrity} is the property that the recipient of a message can detect (with high probability) tampering.
Understanding as tampering, any manipulation during the transmission of the message.

A commonly used way to ensure integrity to a message is appending a \textbf{message authentication code} (MAC) to it.

In a MAC scheme a long term secret key $k$ is shared between a sender and a recipient.
This key is used in two operations, the first being generating the tag or MAC.
The second is the verification with a given pair message and tag that the message is not modified.
Both oprations must follow Equation~\ref{eq:mac_prot} in order to ensure correctness.

\begin{equation}
  Ver(k, m, MAC(k, m)) = 1
  \label{eq:mac_prop}
\end{equation}

The authentication procedure starts with the sneder generating a tag $t$ (see Equation~\ref{eq:mac_gen}).
The sender then sends to the recipient the pair $(m, t)$ through an insecure channel.
The recipient receives a pair $(m', t')$ that can be different from the sent pair.
The recipient can then check the integrity by using the comparison in Equation~\ref{eq:mac_ver}.

\begin{equation}
  t = MAC(k, m)
  \label{eq:mac_gen}
\end{equation}

\begin{equation}
  Ver(k, m) ?= 1
  \label{eq:mac_ver}
\end{equation}

MAC could be a probabilistic algorith, meaning there would be several valid tags for a given message and key.
However, normally it is deterministic and the verification function just recalculates the tag and compares it with the received.
In this case, correctness is trivial.

A MAC system is secure if an attacker is unable to forge a valid pair of message and tag.
But, like in cryptography, different attack scenarios can be defined, giving several ressources and goals to the attacker.

In the simplest case the attacker knows the algorithm and some valid pairs from messages recorded.
The attacker wins if he is able to recover the key.
Other goals could be \textbf{universal forgery}, the attacker can forge a valid tag for any message without the key.
Another could be \textbf{existential forgery}, the attacker can forge a tag for a particular message chosen by him.
This message is not within the known message and tag pairs.
The ressources for this attack can be some valid pairs for messages chosen by the attacker or not.

A correct label attached to a message not only gives the guarantees that the massage has not been modified.
It also ensure that the sender computed the tag, as he is the only one other than the recipient knowing the key.

Perfect schemes don't exist as with enough message and tag pairs most are broken.
For this reason, in security proofs we limit the amount of pairs available.

    \subsubsection{Information Theoric MAC}{
      If we define the following MAC scheme.

\begin{equation*}
  k = (a, b)
\end{equation*}

\begin{equation*}
  MAC(k, m) = am + b
\end{equation*}

\begin{equation*}
  Ver(k, m, t): am + b = t?
\end{equation*}

If we limit the attacker to knowing at most one valid pair, even if the message is of his choosing we can prove that the probability of a valid forgery is $1/q$.
Indeed, valid pairs are points on a line in the plane. Therefore, guessing another point in the line is essentially guessing the slope $a$.

The MAC is called \textbf{information theoretically secure} because the probability $1/q$ is the same as the success probability of a brute-force attack guessing the tag from scratch.
However, if the attacker knows two different valid pairs then it can launch a successful key recovery attack.
For this reason, it is known as a one-time information theoric MAC.

This idea can be extended to a n-times information theoric MAC.
We simply exten the degree of the polynomial from 1 to n.
Then the attacker can know at most n valid pairs before performing a key recovery attack.

This scheme is not practical for several reasons.
Firstly, it requires a key n times as long as the message.
Additionally, the tag is as long as the message.
Practical MACs are more efficient than this one.

    }
    \subsubsection{Block Cipher MAC}{
    }
  }
  \subsection{Hash functions}{
    \subsubsection{Random Oracle}{
    }
    \subsubsection{Practical Constructions}{
    }
    \subsubsection{Applications}{
    }
  }
  \subsection{Authenticated Encyption}{
    \subsubsection{Encrypt-then-MAC Generic Construction}{
    }
    \subsubsection{Particular Constructions}{
    }
  }
}
\pagebreak

\section{Terminology}{
	In this document, the key words "MUST", "MUST NOT", "REQUIRED",
	"SHALL", "SHALL NOT", "SHOULD", "SHOULD NOT", "RECOMMENDED", "MAY",
	and "OPTIONAL" are to be interpreted as described in BCP 14, RFC 2119   [RFC2119].
	
	\paragraph{stuff} Todo
}
\pagebreak

\section{Overview}{
	\lipsum[4]
}

\end{document}
